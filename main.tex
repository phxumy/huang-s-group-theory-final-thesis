% !TeX program = xelatex
\documentclass[UTF8,a4paper,12pt]{ctexart}


% ======================
% 版式与常用宏包
% ======================
\usepackage[margin=2.3cm]{geometry}
\usepackage{graphicx}
\usepackage{caption}
\usepackage{subcaption}
\usepackage{booktabs}
\usepackage{amsmath,amssymb}
\usepackage{enumitem}
\usepackage{hyperref}
\usepackage{cleveref}
\usepackage{setspace}
\usepackage{tcolorbox}
\usepackage{xcolor}
\usepackage[backend=biber,style=numeric]{biblatex}
\addbibresource{refs.bib}
\hypersetup{
  colorlinks=true,
  linkcolor=blue,
  urlcolor=blue,
  citecolor=blue
}
% 让标题/目录里出现乘号也不容易触发 hyperref 的 PDF string 警告
\newcommand{\mulA}{\texorpdfstring{$3\times 8$}{3×8}}
\newcommand{\mulB}{\texorpdfstring{$8\times 3$}{8×3}}

\setlist[itemize]{itemsep=0.2em, topsep=0.2em}
\setlist[enumerate]{itemsep=0.2em, topsep=0.2em}

% 行距:小论文一般 1.25-1.4 比较舒服
\setstretch{1.28}

% 图片默认路径
\graphicspath{{figures/}}

% ======================
% 自定义:信息栏/关键词/观点框
% ======================
\newcommand{\Course}{群论(课程作业小论文)}
\newcommand{\School}{(可选:学校/学院)}
\newcommand{\AuthorName}{(你的名字)}
\newcommand{\StudentID}{(学号,可选)}
\newcommand{\DateText}{\today}

\newcommand{\keywords}[1]{\noindent\textbf{关键词:}#1}

\newtcolorbox{opinionbox}{
  colback=gray!8,
  colframe=gray!45,
  arc=2mm,
  left=1.2mm,right=1.2mm,top=1.0mm,bottom=1.0mm
}

% ======================
% 标题信息
% ======================
\title{\bfseries 从乘法交换律到“对称的思想”\\——初等教育中适当引入群概念的可能性与“$3\times 8$ vs $8\times 3$”之争}
\author{
  徐铭禹\\
  中国科学院大学物理科学学院;物理研究所\\
  {\small \texttt{xumingyu251@mails.ucas.ac.cn}}
}
\date{\DateText}

\begin{document}
\maketitle

\vspace{-0.5em}
\begin{center}
\small
课程:群论 \quad
学号:202528000807041 \quad
\end{center}
\vspace{0.3em}

% ======================
% 摘要
% ======================
\begin{abstract}
群论是一个研究“带有运算的对象结构”的数学分支,其核心思想之一是“通过考察对象的对称性与运算规律,理解其内在结构”。
2025年网络上曾经出现过新修订的小学二年级教材中$3\times 8$与$8\times 3$不等价的争论,
在学习了群论这门课程之后,笔者对这一争论有一些思考,本文中会判断该争论在不同语境下是否有意义,并且讨论在初等教育中适当引入群论知识的可能性。
本文的核心观点是:在不引入形式化公理的前提下,群论可以作为‘结构与对称’的直觉训练进入小学;而关于乘法顺序的争论,若停留在算术等式层面意义不大,但在建模与单位解释层面具有教学价值。
\end{abstract}

\keywords{群论思想;小学数学;对称;乘法交换律;}

\tableofcontents
\newpage

% ======================
\section{引言:“3×8 和 8×3”的争论}
根据中国青年网\cite{web:commutative}的报道,一道小学二年级的数学题问题是有3个盘子,每盘8个水果,一共有多少个水果?列式是3×8还是8×3?
孩子写了3×8,却被老师判错。
老师判错的依据是写乘法算式,先找每份数写在前面,再找有几份写在后面。
看到这个新闻我的第一个反应是一个法国笑话。引用自\cite{web:joke}
\begin{opinionbox}
一个法国人问一个小学生:你知道2+3等于多少吗?\\
小学生:不知道。\\
法国人:那3+2呢?\\
小学生:不知道。\\
法国人:那你在学校都学了什么?\\
小学生:我知道3+2等于2+3。\\
法国人:为什么?\\
小学生:因为这是一个阿贝尔群。
\end{opinionbox}
这实际上是一个教育学的笑话。
历史上曾有以法国一些学者为首的数学家主张从底层的数理逻辑和公理化开始教学数学,例如从皮亚诺公理而不是直观来引入自然数。
这种教学方式虽然严谨,但并不利于初学者的理解,其反对者制作了这个笑话来说明这种教学方式的荒谬。
这种教育学思潮是所谓的布尔巴基学派(Nicolas Bourbaki)带来的\cite{web:Bourbaki},这个团体于20世纪30年代开始形成。
布尔巴基是这个团体的成员已出版的《数学原理》(约40卷)一书作者的笔名。
他们以结构主义观点从事数学分析,认为数学就是关于结构的科学、数学结构没有任何事先指定特征,它是只着眼于它们之间关系的对象的集合。
在各种数学结构之间有其内在的联系,其中代数结构、拓扑结构和序结构是最基本的结构,称为母结构,而其他结构则是由较为基本的结构交叉、复合而生成的子结构。
由于布尔巴基曾获得很大成就,使得“新数学”从60年代起就进入中小学数学教学,从而造成了巨大的社会问题。幼稚园的小朋友要学集合论,到中学就要教环与理想,这不仅学生吃不消,连教师也叫苦连天。这种“新数学”教育在法国、美国等国家推行一段时期后,效果明显不佳,因此有些人就迁怒于布尔巴基,形成了一股反布尔巴基的浪潮。
网上也流传着经典梗图来讽刺这一历史事件。

\begin{figure}[!htbp]
\centering
\includegraphics[width=0.73\linewidth]{new-math.png}
\caption{“最新版小学数学大纲”}
\label{fig:new-math}
\end{figure}

似乎稍微有点跑题,总之,这段历史和3×8与8×3的争论站在了两个对立面上。
假如新闻中的这名教师确实接到的是这样的教学通知的话,可以发现我国目前最新版的小学数学教学过于强调数学与现实生活的联系,也就是过于实体化并且不加解释地为小学生做了这么个好像规则怪谈一样的法则,
然而这种法则实际上在他接受到了更高级的教育之后是毫无意义的,很可能小孩就在这种无意义的法则的记忆与运用中丧失了对学习的兴趣;而“布尔巴基学派”则过于抽象化。

其实从一个物理学研究生的角度来说,我认为这则新闻中的争论很好解决,一个居中的办法就是早点引入量纲分析的概念:
比如每人三个苹果,一共有八个人,那么总共有多少苹果?
这个问题的答案是3个苹果/人×8人=24苹果,自然也可以写成8人×3苹果/人,而不是8苹果×3人=24苹果·人,或者什么其他的单位。
这样既让学生理解了数学结构上的交换律,也没有丧失份数和每份数量的语义。
在我的记忆中,这种实际运算是可以带单位,并且单位之间也可以乘除的思想直到初中学习物理才开始接触。
并且很多老师也是讲的语焉不详。
可能也是地区教育差异的原因,我清楚地记得下面这个道理是我小学时自己悟出来:原来3个/人中的斜杠,是“每”的意思,他和除法的除号、分数中间的横杠是等价的。

其实大多数人都回答不上这样一个问题:“2+3到底为什么等于5?”如果真要完完整整地回答好这个问题,可能还真要从皮亚诺公理开始讲起,
所以我并不反对使用数小棍或者拿苹果举例这种方式来在小学生心中建立起“数”的概念。
但我的建议是小学生多少已经具有了一些抽象化的思维,这种举例可以作为引入手段但并不应该成为桎梏,可以适当在初等教育中引入一些结构化的群论思想,来帮助学生理解数学结构与现实生活的联系,只要别把小学大纲改成图\ref{fig:new-math}那样就好。
\section{背景:群论到底提供了什么“思想工具”?}
群论的一个关键贡献不是“算出答案”,而是提供一种\textbf{用运算与对称刻画对象}的方式。通俗地说:我们不先问对象是什么,而先问“可以对它做哪些操作、这些操作如何组合”。
记得本科群论的第一节课,老师就跟我们说,群论不涉及到什么高级的数学工具,会加减乘除的小学生也能学(现在想来要除去群表示论)。
\subsection{“群”的四个条件}
群的成立概念当然也可以用非数学语言的方式描述:
群就是一些操作的集合,这些操作要满足可做且做完还在同一类里,即封闭性;
先后做三步,括号怎么加都一样,即结合律;有个“什么都不改变”的操作,即单位元;
每个操作都能撤销,即逆元。
如果一定要给出最短的形式化,可以写成(但这种自然不适合让小学生理解):
\[
(G,\circ)\ \text{满足封闭性、结合律、单位元、逆元。}
\]

\subsection{小学初中可接触的“群论前置经验”}
群论的思想其实在小学教学中很常见。
就仅限于我们这学期学的有限群和少部分李群的内容范围,小学阶段的学生其实已经能接触到一些“群论影子”了。
\begin{table}[h]
\centering
\caption{小学初中情境中的“群论影子”}
\begin{tabular}{@{}lll@{}}
\toprule
知识内容 & 操作集合(直觉) & 可对应的群论关键词 \\ \midrule
平面图形对称 & 旋转、轴对称、中心对称 & 对称变换、空间群 \\
钟表问题 & 加 1 小时、加 2 小时……循环 & 模运算、循环结构 \\
排队换位 & 交换两个人、轮换位置 & 置换、对换 \\ \bottomrule
\end{tabular}
\end{table}

\section{初等教育阶段“引入群概念”的可行方式}
这个引入最好也要和实际生活相关联,比如魔方的转动、钟表的时间计算、小朋友换座位问题。
因为笔者并非教育学专业,所以这里抛砖引玉,给一些简单的课堂活动设计思路,供大家参考。
比如旋转一次魔方就是一次操作,再旋转一次就相当于两个操作相乘,可以通过实际操作让学生体会“操作的组合”。
并且可以让他们理解操作的先后顺序对结果的影响,从而引入“非交换群”的概念。
再例如钟表的旋转,可以让学生体会“模 12 运算”,从而理解“循环群”的概念。
小朋友换座位问题,则可以让学生体会“置换群”的基本概念。、
当然,课堂引入这些不是让学生学习公理化的定义,而是让他们一定程度上理解结构化的数学。


\section{结论与展望}
本文围绕“是否能在小学教育中适当引入群的概念”以及 \mulA\ vs \mulB\ 的争论给出了一个结构化回答。首先,群论作为课程内容当然不应直接下沉到小学,但群论所代表的\textbf{结构与对称}的思想可以通过活动进入小学:让学生在“可重复的操作、操作的组合、可逆与分类、对称与不变性”的体验中形成直觉。这种直觉不是形式主义负担,反而能降低日后学习抽象数学时的断裂感。

其次,\mulA\ 与 \mulB\ 在算术层面等价,但在建模与表达规范层面未必等价。若课堂只追求计算正确性,则不应把交换顺序判为错误;若课堂希望训练“把语言情境转为算式”的能力,则在某些阶段强调固定的语义分解是合理的。真正需要避免的是“只给规则不给解释”:当规则与交换律这一结构事实发生张力时,学生最容易在机械记忆中丧失兴趣。

最后,一个更具建设性的路径是:比起争论“哪个式子对”,不如更早、以更直观的方式引入“单位/量纲一致性”的观念,让学生自然理解 $3\ \text{盘}\times 8\ \text{个/盘}$ 这类表达为什么有意义、交换后又该如何解释。未来若要把本文观点落到实处,可以设计小规模课堂实验:比较“允许多种列式但要求解释一致”与“强制唯一列式规范”两种策略对学生长期理解与兴趣的影响,从而把争论从口水战转化为可检验的教学改进。



\printbibliography

\end{document}
